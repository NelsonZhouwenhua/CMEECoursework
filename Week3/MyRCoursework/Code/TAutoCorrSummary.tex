\documentclass{article}
\usepackage[utf8]{inputenc}

\title{TAutoCorr.R Summary}
\author{Wenhua Zhou(wz2812@ic.ac.uk)}
\date{October 2019}

\usepackage{natbib}
\usepackage{graphicx}
\usepackage{listings}
\usepackage[svgnames]{xcolor}

\begin{document}

\maketitle

\section{Original code}
\lstset{language=R,
    basicstyle=\small\ttfamily,
    stringstyle=\color{DarkGreen},
    otherkeywords={0,1,2,3,4,5,6,7,8,9},
    morekeywords={TRUE,FALSE},
    deletekeywords={data,frame,length,as,character},
    keywordstyle=\color{blue},
    commentstyle=\color{DarkGreen},
    frame=single,
    numberstyle=\tiny,
    numbersep=5pt,
    breaklines=true
}
\begin{lstlisting}[caption=Source code for TAutoCorr.R(including comments)]
# R practicals
# read data and compute correlation and p-value

# read data from KeyWestAnnualMeanTemperature.Rdata
load(file = "/home/nelson/Documents/CMEECoursework/Week3/MyRCoursework/Data/KeyWestAnnualMeanTemperature.RData")

# correlation
correlation <- cor(ats)[2,1]

# function of compute randomly permuting time series and calculate correlation
random_correlation <- function(x){
  a <- sample(x[,1],100) # random sample the 100 years of time series
  b <- cbind(a,x[,2]) # combine sampled time series and original temperature
  return(cor(b)[2,1]) # return the calculated correlation
}

# calculate the random correlation 10000 times
RandCor <- sapply(1:10000, function(i) random_correlation(ats)) 

# calculate the fraction of the correlation coefficients(p-value)
p <- sum(RandCor > correlation)/10000
\end{lstlisting}

\begin{lstlisting}
> correlation
[1] 0.5331784
> p
[1] 3e-03
\end{lstlisting}

\section{Results and conclusion}
From the code above, the correlation calculated by the given R data is about 0.326. This number illustrates that there is likely to be a correlation between the temperature of two successive years. We can see that more precisely by looking at another output variable, the p-value is 0.0003 in this situation. In particular from our code, this p-value means if we randomly permute time series 10000 times, there is only 3 times the random correlation will be bigger than the original correlation of the data. Generally if p-value is smaller than 0.05, there is a correlation between the temperature of two successive years. Therefore we can conclude that the temperature is positive correlated to the temperature one year before.




\end{document}
